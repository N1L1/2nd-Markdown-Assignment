% Options for packages loaded elsewhere
\PassOptionsToPackage{unicode}{hyperref}
\PassOptionsToPackage{hyphens}{url}
%
\documentclass[
]{article}
\usepackage{amsmath,amssymb}
\usepackage{lmodern}
\usepackage{ifxetex,ifluatex}
\ifnum 0\ifxetex 1\fi\ifluatex 1\fi=0 % if pdftex
  \usepackage[T1]{fontenc}
  \usepackage[utf8]{inputenc}
  \usepackage{textcomp} % provide euro and other symbols
\else % if luatex or xetex
  \usepackage{unicode-math}
  \defaultfontfeatures{Scale=MatchLowercase}
  \defaultfontfeatures[\rmfamily]{Ligatures=TeX,Scale=1}
\fi
% Use upquote if available, for straight quotes in verbatim environments
\IfFileExists{upquote.sty}{\usepackage{upquote}}{}
\IfFileExists{microtype.sty}{% use microtype if available
  \usepackage[]{microtype}
  \UseMicrotypeSet[protrusion]{basicmath} % disable protrusion for tt fonts
}{}
\makeatletter
\@ifundefined{KOMAClassName}{% if non-KOMA class
  \IfFileExists{parskip.sty}{%
    \usepackage{parskip}
  }{% else
    \setlength{\parindent}{0pt}
    \setlength{\parskip}{6pt plus 2pt minus 1pt}}
}{% if KOMA class
  \KOMAoptions{parskip=half}}
\makeatother
\usepackage{xcolor}
\IfFileExists{xurl.sty}{\usepackage{xurl}}{} % add URL line breaks if available
\IfFileExists{bookmark.sty}{\usepackage{bookmark}}{\usepackage{hyperref}}
\hypersetup{
  pdftitle={Second Markdown Assignment},
  hidelinks,
  pdfcreator={LaTeX via pandoc}}
\urlstyle{same} % disable monospaced font for URLs
\usepackage[margin=1in]{geometry}
\usepackage{longtable,booktabs,array}
\usepackage{calc} % for calculating minipage widths
% Correct order of tables after \paragraph or \subparagraph
\usepackage{etoolbox}
\makeatletter
\patchcmd\longtable{\par}{\if@noskipsec\mbox{}\fi\par}{}{}
\makeatother
% Allow footnotes in longtable head/foot
\IfFileExists{footnotehyper.sty}{\usepackage{footnotehyper}}{\usepackage{footnote}}
\makesavenoteenv{longtable}
\usepackage{graphicx}
\makeatletter
\def\maxwidth{\ifdim\Gin@nat@width>\linewidth\linewidth\else\Gin@nat@width\fi}
\def\maxheight{\ifdim\Gin@nat@height>\textheight\textheight\else\Gin@nat@height\fi}
\makeatother
% Scale images if necessary, so that they will not overflow the page
% margins by default, and it is still possible to overwrite the defaults
% using explicit options in \includegraphics[width, height, ...]{}
\setkeys{Gin}{width=\maxwidth,height=\maxheight,keepaspectratio}
% Set default figure placement to htbp
\makeatletter
\def\fps@figure{htbp}
\makeatother
\setlength{\emergencystretch}{3em} % prevent overfull lines
\providecommand{\tightlist}{%
  \setlength{\itemsep}{0pt}\setlength{\parskip}{0pt}}
\setcounter{secnumdepth}{-\maxdimen} % remove section numbering
\ifluatex
  \usepackage{selnolig}  % disable illegal ligatures
\fi

\title{Second Markdown Assignment}
\author{}
\date{\vspace{-2.5em}}

\begin{document}
\maketitle

\hypertarget{the-monty-hall-problem}{%
\subsubsection{\texorpdfstring{\textbf{The Monty Hall
Problem}}{The Monty Hall Problem}}\label{the-monty-hall-problem}}

\hypertarget{briefly-discuss-the-monty-hall-problem-and-share-your-insights-and-solution-to-this-brain-teaser-problem.}{%
\paragraph{\texorpdfstring{ Briefly discuss the Monty Hall problem and
share your insights and solution to this brain teaser
problem.}{ Briefly discuss the Monty Hall problem and share your insights and solution to this brain teaser problem.}}\label{briefly-discuss-the-monty-hall-problem-and-share-your-insights-and-solution-to-this-brain-teaser-problem.}}

~~~~~~~~~~The \textbf{Monty Hall problem} is a probability puzzle named
after the late Monty Hall, the original host of the TV show Let's Make a
Deal. In this problem, there are \emph{3 doors}. Behind 1 door is
\emph{a car}; the grand prize that the contestant can win, while behind
the other 2 doors contain \emph{goats}. The contestant chooses a door at
random which he or she believes is the door that contains the grand
prize. The game host then opens one of the 2 remaining doors which
reveals one of the goats inside. Now the contestant has the option to
stick with his or her chosen door or switch to the other or remaining
door to win the prize. The majority of people assume that both doors
have the same probability to win, where the door you chose and the door
that was left unopened have a 50/50 chance. Because there is no
perceived reason to change, most stick with their initial choice.
However, in terms of chances, that is not the case.

\begin{figure}
\centering
\includegraphics{https://external-preview.redd.it/9j6tqnfWlm44g9GjPSJI29X7fbPzzc7kZKa0GvcsyM0.jpg?width=640\&crop=smart\&auto=webp\&s=62a58e68994fd8233507379ab4aafad41e84b758}
\caption{Monty Hall Problem}
\end{figure}

~~~~~~~~~~The chances of actually winning the car is, at first,
distributed equally among each door, meaning that each door will have a
⅓ chance of containing the grand prize. If a contestant chooses 1 door,
he or she will have a ⅓ chance of having the grand prize, and it would
stay that way if he or she sticks with his or her chosen door. That
would mean that there must be a ⅔ chance that the car is on the other 2
doors. However, even though the host had already revealed the contents
of the second door, the contestant would still have a ⅔ chance of
winning the grand prize if he or she switches. The odds are better if
you switch because the host curates the remaining choices (Statistics
How to, 2021).

\begin{figure}
\centering
\includegraphics{https://i0.wp.com/www.geeksaresexy.net/wp-content/uploads/2010/05/Monty_open_door_chances.svg_.png?resize=500\%2C508}
\caption{Monty Hall Problem Probability rates}
\end{figure}

~~~~~~~~~~To better understand the concept of probability in this game,
let us look into the answer of columnist Marilyn vos Savant in her
magazine Parade when she was asked about the same problem. According to
her, there is about 66\% chance (roughly ⅔) of winning the prize by
switching (Frost, 2021). This is because there are only nine different
combinations of choices and outcomes. Let us refer to this table:

\begin{verbatim}
##    row1 row2               row3               row4
## 1     1    1                Win               Lose
## 2     1    2               Lose                Win
## 3     1    3               Lose                Win
## 4     2    1               Lose                Win
## 5     2    2                Win               Lose
## 6     2    3               Lose                Win
## 7     3    1               Lose                Win
## 8     3    2               Lose                Win
## 9     3    3                Win               Lose
## 10           3 Wins (About 33%) 6 Wins (About 66%)
\end{verbatim}

\begin{longtable}[]{@{}cccc@{}}
\toprule
Contestant's initial door & Door with prize & Don't Switch & Switch \\
\midrule
\endhead
1 & 1 & Win & Lose \\
1 & 2 & Lose & Win \\
1 & 3 & Lose & Win \\
2 & 1 & Lose & Win \\
2 & 2 & Win & Lose \\
2 & 3 & Lose & Win \\
3 & 1 & Lose & Win \\
3 & 2 & Lose & Win \\
3 & 3 & Win & Lose \\
& & 3 Wins (About 33\%) & 6 Wins (About 66\%) \\
\bottomrule
\end{longtable}

~~~~~~~~~~The first column contains the numbers 1, 2 and 3 which
represent each of the doors or labels each door. Meanwhile, the second
column represents the door that has the prize. The third and fourth
column tells whether the contestant wins or loses the game if he or she
either sticks with his or her first choice or switches to the other door
respectively. By enumerating these nine different combinations of
choices and outcomes, we now have different potential situations where
we could analyze the probability of winning between switching and not
switching the contestant's first choice. In the third column which
represents the outcome if the contestant didn't switch, most of the
result would indicate that the contestant would have a 33\% chance of
winning if he or she sticks with his or her first choice; having a total
of 3 wins and 6 losses considering all 9 scenarios. Meanwhile, if the
contestant switches his or her door for the other unopened door, then he
or she would have a bigger chance to win; having a total of 6 wins and 3
losses considering all 9 scenarios, which gives him or her a 66\% chance
of winning. If we compare the probabilities of winning between switching
and not switching the door, then from the probability percentages we
could conclude that the contestant would usually have a better chance of
winning the prize if he or she switches his or her choice.

~~~~~~~~~~~~Still not convinced that switching would give us a better
chance of winning? Then let's try increasing the number of doors, from a
measly 3 to a hundred. Each door would now have 1/100 chance of
containing the prize. The contestant still picks one door, but the host
opens 98 doors so that the contestant would have the option to either
stick with the door he or she initially chose or switch to the other. If
we always keep our original door, then our probability of getting the
car is the probability we choose the door on the first try, which is
1/100. If we choose to always switch, then our probability of ultimately
getting the car is the probability we choose a goat times the
probability we switch from a goat to a car. Our probability of picking a
goat initially is clearly 99/100. Then, once we pick a goat and one goat
door is opened, there are 98 other doors, one of which contains the
prize. This would mean that our chance of switching from a goat door to
a car door, or the chance of guessing correctly after switching is 1/98,
which also demonstrates sampling without replacement.

Here is the solution for solving the probability of winning the prize by
switching:

Let P(A) be the Probability of getting a goat Let P(B\textbar A) be the
Probability of guessing correctly after switching Let P(A
\emph{intersection} B) be the Probability of getting the prize or
winning by switching

\[ P(A)= \frac{99}{100} \] \[ P(B|A) = \frac{1}{98}\]
\[ P(A \bigcap B) = P(A)*P(B|A) = \frac{99}{100} * \frac{1}{98} \approx 0.0101 \]

~~~~~~~~~~Now let's compare that to the probability of getting the prize
on the first try, which is 1/100 or approximately 0.01. You'll notice
that the Probability of getting the prize or winning by switching is
\textbf{slightly greater} than the probability of getting the prize on
the first try of sticking with the initially chosen door (0.0101
\textgreater{} 0.1). This would still indicate that we would usually
have a higher chance of getting the prize if we ultimately switch from
our first choice rather than sticking with our first choice.

\textbf{Names: Justine Sison and Sun Phil Zablan}

\textbf{References:}

Frost, J. (2021, February 01). The monty hall problem: A statistical
illusion. Retrieved July 15, 2021, from
\url{https://statisticsbyjim.com/fun/monty-hall-problem/}

Statistics How to (2021, June 13). Monty Hall problem: Solution
explained simply. Retrieved July 15, 2021, from
\url{https://www.statisticshowto.com/probability-and-statistics/monty-hall-problem/}

\begin{center}\rule{0.5\linewidth}{0.5pt}\end{center}

\hypertarget{the-psco-lottery}{%
\subsubsection{\texorpdfstring{\textbf{The PSCO
Lottery}}{The PSCO Lottery}}\label{the-psco-lottery}}

\hypertarget{one-of-the-most-popular-major-games-of-pcso-is-the-ultra-lotto-658.-at-20-per-ticket-how-much-would-a-bettor-spend-to-cover-all-of-the-possible-combinations.-would-the-grand-prize-of-50-million-cover-all-the-expenses-or-it-would-simply-incur-a-massive-loss-on-the-bettor}{%
\paragraph{\texorpdfstring{ One of the most popular major games of PCSO
is the Ultra Lotto 6/58. At ₱20 per ticket, how much would a bettor
spend to cover all of the possible combinations. Would the grand prize
of ₱50 million cover all the expenses or it would simply incur a massive
loss on the
bettor?}{ One of the most popular major games of PCSO is the Ultra Lotto 6/58. At ₱20 per ticket, how much would a bettor spend to cover all of the possible combinations. Would the grand prize of ₱50 million cover all the expenses or it would simply incur a massive loss on the bettor?}}\label{one-of-the-most-popular-major-games-of-pcso-is-the-ultra-lotto-658.-at-20-per-ticket-how-much-would-a-bettor-spend-to-cover-all-of-the-possible-combinations.-would-the-grand-prize-of-50-million-cover-all-the-expenses-or-it-would-simply-incur-a-massive-loss-on-the-bettor}}

\begin{figure}
\centering
\includegraphics[width=0.1\textwidth,height=\textheight]{https://scontent.fmnl4-5.fna.fbcdn.net/v/t1.6435-9/74602780_3029414650418286_6084237558978445312_n.png?_nc_cat=106\&ccb=1-3\&_nc_sid=730e14\&_nc_eui2=AeFxiD0eRGZjIo5xHm1JZugSW5UF5UUcNKJblQXlRRw0okZi5JSfg645-NqYxRqcR1poE0cNUo5gEh8PdOombRXR\&_nc_ohc=OG6n8o3lQkMAX-wBHEM\&_nc_ht=scontent.fmnl4-5.fna\&oh=29f8b6de6f1efeae6d830e4fdf02678f\&oe=60F48579}
\caption{PCSO Logo}
\end{figure}

The \textbf{PCSO Lottery Draw} is \emph{a television game show produced
by the Philippine Charity Sweeptakes Office (PCSO)} that started airing
since March 8, 1995 on People's Television Network Channel (PTV). The
program consists of drawing of parimutuel and fixed payout lottery
games, sweepstakes games such as Lotto 6/42, Mega Lotto 6/45, Super
Lotto 6/49, Grand Lotto 6/55, Ultra Lotto 6/58, and other fixed payout
games.

The PCSO runs a variation of the lottery with 6 numbers, all from one to
fifty-eight, which is the Ultra Lotto 6/58 event. The rules of the PCSO
are simple, if the set of numbers picked by the host matches the numbers
on your ticket, you win a prize pool of money! Take note the more
winners there are, the more your prize pool is split.

This sounds like a lucrative to make money because of the massive prize
pool and possible profit you gain from a single ticket, which costs ₱20.
But with the lotto being the lotto, the odds are guaranteed to be
stacked against everyone who participates, including you. So in this
problem, we will find out just how much money it will take to have a
100\% chance of finding the lucky number.

\textbf{A few important points about the PCSO lotto if you have never
played it before:}

Six number balls will be picked from 1 -- 58 through a vacuum system,
assuring unpredictability

Once a ball is taken out, it is removed from the pool of possible
numbers

The order of the numbers do not matter

These points warrant the usage of the Combination Formula:

\[ _{n}C_{k}= \frac{n!}{n!(n-k)!} \] Where \textbf{\emph{n}} is the
\textbf{total number of possible numbers (58)}

And \textbf{\emph{k}} is the \textbf{number of numbers being picked (6)}

We will be using the \textbf{combination formula} \emph{because the
order does not matter as long as you have the matching numbers to the
ones the machine have picked.} This formula will help us find the total
number of tickets we need to buy to guarantee a chance at winning the
lottery. Inserting our known values into the formula gives us

\[ _{58}C_{6}= \frac{58!}{58!(58-6)!} \] which is equal to: \textbf{40,
475, 358 lottery tickets}. With each ticket costing ₱20, multiplying the
value gained with 20 will give us the total amount of money we need to
spend to guarantee a win.

\[ 40, 475, 358*₱20 = ₱809,507,160\]

Now that is a ton of money. With the theoretical grand prize being
₱50,000,000, let us see how much we will gain by buying that many
tickets for that amount of money.

\[ ₱50,000,000 - ₱809,507,160 = ₱-759,507,160 \] Now I do not know about
you, but this feels like a net loss of astronomical scale.

In conclusion, spending enough money to have a 100\% chance of winning
the lottery would incur a financial loss on the bettor, one they would
never be able to recover from, even if they win the lottery three times
over.

The sheer amount of possible numbers in the lottery and the millions of
bettors ensure its continued activities, meaning the ones running it
will always make sure the odds are stacked against each individual
bettor. In the off-chance that you do win the lottery, just know that
you pulled of something that only happens to one in over forty million
possibilities.

\begin{figure}
\centering
\includegraphics{https://hips.hearstapps.com/hmg-prod.s3.amazonaws.com/images/djdjn-djdjndj\%C2\%B5ndjdj\%C2\%B5ndj-royalty-free-illustration-1592420083.jpg?crop=1xw:1xh;center,top\&resize=640:*}
\caption{Little Probability}
\end{figure}

\textbf{Names: Lester Salvador and Jose Xavier Castillo}

\textbf{Reference:}

Niemi, J. (2020). Dr.~J's Guide to Combinations (without replacement)
{[}Video{]}. In YouTube.
\url{https://www.youtube.com/watch?v=nFz1UX-bZmA}

\begin{center}\rule{0.5\linewidth}{0.5pt}\end{center}

\begin{center}\rule{0.5\linewidth}{0.5pt}\end{center}

\end{document}
